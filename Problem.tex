
\section{Løsninger til teoretiske problemer:}


\subsection*{Problem 1.}

Ligningen (2.3) viser, at x er lig med en kombination af to ting: $\frac{(1-d)}{N}e$ og $d\cdot L\cdot x$. $e$ er en vektor bestående af kun 1'ere, $E_N$ er en matrix bestående af kun 1'ere.
Da $L$ er en kendt matrix, og $d$ er en kendt konstant og $N$ er en kendt heltal, kan vi erstatte e og $E_N$ i ligningen (2.3) for at få :
$$x = \frac{(1-d)}{N}E_N + d\cdot L\cdot x$$
Vi kan derefter definere $M_d$ som $\frac{(1-d)}{N} \cdot E_N + d \cdot L$ og erstatte det i ligningen (2.3) for at få:
$$x = M_d \cdot x$$
Så ligningen (2.3) kan skrives som ligningen (2.4) hvor $$M_d = \frac{(1-d)}{N}E_N + d \cdot L$$
Så vi har vist, at ligningen (2.3) kan skrives om som ligningen (2.4)



\subsection*{Problem 2.}

Bevis for at en Markov matrice med en potens $m$ ligeså er en Markov matrice:

Der er forskellige regler før at en Markov matrice $(A)$ er rigtigt. For det første kan $A$ ikke indeholde negative tal. Derudover skal summen af rækkerne give 1. En Markov matrice skal også være en $n \times n$ matrix. Dette vil sige, at hvis vi tager en Markov matrix $A$ og hæver den til en potens $m$ altså: $A^m$, vil den resulterende matrix også være en Markov matrix for enhver $m = 0,1,2,3,...$

Dette skyldes, at kan hæve en matrix til en potens $m$ simpelthen betyder at multiplicere matricen med sig selv $m$ gange. Derfor vil den resulterende matrix stadig have ikke-negative elementer og hver række vil stadig give summen 1, da disse egenskaber bevares under matrixmultiplikation.

\subsection*{Problem 3.}

Sætning 3.2 siger, at hvis $A \in R^{nxn}$ er en Markov matrix, så er den spektral radius (det største egenværdi i modulus) lig med 1. For at bevise dette, antager vi at der er en egenværdi $\lambda$ med en egenvektor $v$, så $A^kv = \lambda^kv$, hvor k er et positiv heltal. Men da $A^k$ er en Markov matrix, så er summen af hver række = 1, hvilket kan ikke matche med $\lambda^kv$, hvis $|\lambda| > 1$, da $|\lambda^k|$ vokser større som k øges, så det er en modsætning. Så spektral radius kan kun være mindre eller lig med 1.

\subsection*{Problem 4.}
ATT: Mangler lidt.


Hvis $A$ er en Markov matrix og $0 < \tau < 1$, så skal vi bevise at $A_{\tau}$ også er en Markov matrix med kun positive elementer. Det er således fordi at $A$ er en Markov matrix, det betyder at alle elementer er positive og summen af hver række er 1. Når man tager $A_{\tau}$, så er summen af hver række stadig 1, og alle elementer er stadig positive, da man kun er ved at tage potenser af positive tal. Så $A_{\tau}$ er også en Markov matrix med kun positive elementer.

\subsection*{Problem 5.}

Hvis $A$ er en Markov-matrix med stærkt positive elementer ($a_{ij} > 0$ for alle $i$,$j$), kan man bevise at der kun er én egenværdi $\lambda$ med $|\lambda| = 1$, nemlig $\lambda = 1$, og at den tilsvarende egenvektor er $E_1 = span {\mathbf{e}}$. Enhver anden egenværdi har en modulus mindre end 1.

Man kan bevise dette ved at antage at $\lambda$ er en egenværdi med $|\lambda| = 1$ og $\mathbf{v}$ er en tilsvarende egenvektor. Lad $k$ være et index således at $||\mathbf{v}||_\infty = |v_k|$. Så kan man skrive:

$$||\mathbf{v}||_\infty = |\lambda ||v_k| = |\lambda v_k| = |A\mathbf{v_k}|$$

Ved hjælp af trekantens lighed kan man bevise at:

$$||\mathbf{v}||\infty = |A\mathbf{v_k}| \leq \sum{j=1}^n |a_{jk} v_j| \leq \sum_{j=1}^n a_{jk} |v_j|$$

Da matricen $A$ er en Markov-matrix og egenvektoren $\mathbf{v}$ har positive elementer, kan man skrive:

$$||\mathbf{v}||\infty \leq \sum{j=1}^n a_{jk} |v_j| = ||\mathbf{v}||_\infty$$

Dette er kun sandt hvis alle de ligheder er lige. Det betyder at alle elementerne i egenvektoren er ens og egenvektoren er en multiplum af vektoren med alle 1'ere. Derfor er den eneste egenværdi med $|\lambda| = 1$ $\lambda = 1$ og den tilsvarende egenvektor er $E_1 = span {\mathbf{e}}$.

For at vise at andre egenværdier har en modulus mindre end 1, lader vi $\lambda$ være en egenværdi af $A$ således at $|\lambda| > 1$ og $\mathbf{v}$ være en tilsvarende egenvektor. Så har man:

$$||A^n\mathbf{v}||\infty = |\lambda^n||\mathbf{v}||\infty > ||\mathbf{v}||_\infty$$

Dette strider mod at $A$ er en Markov-matrix, hvor alle elementerne i $A^n$ er positive og summen er 1. Derfor skal alle andre egenværdier have en modulus mindre end 1.


\subsection*{Problem 6.}

Sætning 3.4 siger, at for en Markov matrice A med strikte positive værdier, findes der en unik sandsynlighedsmatrice x, således at produktet af A transponeret opløftet til potensen k $(A^t)^k$ nærmer sig en matrice med den samme matrix x gentaget. Denne matrix $x$ er en unik vektor, hvor alle dele er positive og summere til 1.

PageRank algoritmen, som bruges til at rangere vigtigheden af web sider, bygger på idéen om en "tilfældig surfer" der bevæger sig gennem internettet og følger links. Sandsynligheden for at bevæge sig fra en web side til en anden repræsenteres af en Markov matrice, og den stationære fordeling af denne matrice repræsenterer den langsigtede sandsynlighed for at være på hver side. PageRank algoritmen bruger den stationære fordeling af Markov matricen til at rangere vigtigheden af web siderne, med sider der har en højere sandsynlighed i den stationære fordeling som er betragtes som mere vigtige.

I korthed, sætning 3.4 er relevant for PageRank algoritmen fordi den giver en matematisk begrundelse for at bruge den stationære fordeling af en Markov matrice til at rangere vigtigheden af web sider i algoritmen. Derudover giver sætningen en måde at beregne den stationære fordeling af matricen, som er afgørende for algoritmen.

\newpage
\subsection*{Problem 7.}

Sætning 3.4 i bygger på at $A$ har $n$ egenværdier, bygger på det faktum, at når $A$ har bestemte egenværdier, så har $A$ og dets transponering $A^t$ samme egenværdier. Først ved vi, at alle egenværdierne for $A^t$ har værdier mindre end 1 undtagen én egenværdi, som er lig 1. Derfor vil matricen $(A^t)^k$ nærme sig egenvektorens tilhørende egenværdi 1. Man lader så x være en egenvektor, der ikke er nul. Derfor er $A^t$ tilknyttet egenværdien 1. Hvilket vil sige at vi så har $A^t x = x$. Da matricen $(A^t)^k$ nærmer til egenvektorens tilhørende egenværdien 1, har vi:
$$\lim_{k \to \infty} (A^t)^k = x\mathbf{e}^t = [x, x, . . ., x]$$
Det vil altså sige at summen af vektoren $x$ er lig 1.







